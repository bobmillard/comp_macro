\documentclass{article}

\title{Giovanni Gallipoli Meeting.}
\author{By Robert Millard\footnote{Ph.D. Candidate, The University of Western Ontario.}\\
Working Paper [Please do not cite or circulate]. }

\usepackage{amsmath}
\usepackage{setspace}
\usepackage{mathtools}
\usepackage{fullpage}
\usepackage{graphicx}
\usepackage{mathbbol}
\usepackage{bm}
\usepackage{mathrsfs}
\usepackage{xcolor}
\usepackage{subfloat}
\usepackage{rotating}
\usepackage{adjustbox}
\usepackage[normalem]{ulem}
\usepackage{tikz}
\usetikzlibrary{calc}
\useunder{\uline}{\ul}{}

\setlength{\parindent}{0pt}
\usepackage[normalem]{ulem}
\useunder{\uline}{\ul}{}
\pdfmapfile{+sansmathaccent.map}

\usepackage{booktabs}
\newcommand{\ra}[1]{\renewcommand{\arraystretch}{#1}}


\newcommand\independent{\protect\mathpalette{\protect\independenT}{\perp}}
\def\independenT#1#2{\mathrel{\rlap{$#1#2$}\mkern2mu{#1#2}}}


\graphicspath{ {Images/} }

\begin{document}
%%%%%%%%%%%%%%%%%%%%%%%%%%%%%%%%%%%%%%%%%%%%%%%%%%%%%%%%%%%%%%%%%%%%%%%
\section*{Question 1}
A competitive equilibirum is a set of prices, $\{w_{t}, r_{t}\}_{t=0}^{\infty}$ and allocations $\{c_{t}, k_{t+1}, l_{t} \}_{t=0}^{\infty}$ such that 
\begin{enumerate}
\item Given prices, $\{w_{t}, r_{t}\}_{t=0}^{\infty}$ , the allocation, $\{c_{t}, k_{t+1}, l_{t} \}_{t=0}^{\infty}$, solves
\begin{align}
&\underset{c_{t}, k_{t+1}}{max} \underset{t=0}{\overset{\infty}{\sum}} \beta^{t} u (c_{t})\\
\text{s.t. }& c_{t} + k_{t+1} \leq r_{t}k_{t} + w_{t}l_{t}, \forall t\\
& c_{t}, k_{t}, l_{t} \geq 0, k_{0} \text{ given}.
\end{align}

\item Given prices, $\{ w_{t}, r_{t}\}_{t=0}^{\infty}$ , the allocation $\{k_{t+1}, l_{t} \}_{t=0}^{\infty}$, solves 
\begin{align}
&\underset{l_{t}, k_{t}}{max}\underset{t=0}{\overset{\infty}{\sum}} (y_{t} - w_{t}l_{t} - r_{t}k_{t})\\
\text{s.t. }& y_{t} =  F(k_{t},l_{t}) \\
&y_{t}, k_{t}, l_{t} \geq 0
\end{align}
\item Markets clear: $l_{t} = 1$ and $c_{t} + k_{t+1} = F(k_{t}, 1)$
\end{enumerate}


%%%%%%%%%%%%%%%%%%%%%%%%%%%%%%%%%%%%%%%%%%%%%%%%%%%%%%%%%%%%%%%%%%%%%%%%
\section*{Question 2}
\begin{align}
&\underset{c_{t}, k_{t+1}}{max} \underset{t=0}{\overset{\infty}{\sum}} \beta^{t} u (c_{t}) \\
\text{ s.t. } &k_{t+1} = F(k_{t},1) - c_{t},\\
 &k_{t} \geq 0, c_{t} \geq 0,  k_{0} \text{ given},  
\end{align}

%%%%%%%%%%%%%%%%%%%%%%%%%%%%%%%%%%%%%%%%%%%%%%%%%%%%%%%%%%%%%%%%%%%%%%%%
\section*{Question 3}
As leisure is not in the utility function, consumers allocate all time to work, $l_{t} = 1$, for the competive equilibrium and the social planners problem. \\
$~$\\
Competitive Equilibrium:
\begin{enumerate}
\item Firms problem: Sub in production constraint and take the FOC with respect to capital in period t to obtain
\begin{align}
r_{t} =  F_{k}(k_{t},1)
\end{align}
\item Consumers problem: Sub in the constraint for consumption and take FOC with respect to capital in period t to obtain
\begin{align}
u_{c}(c_{t}) = \beta u(c_{t+1})r_{t}
\end{align}
Now sub in (10) for $r_{t}$ to obtain
\begin{align}
u_{c}(c_{t}) = \beta u(c_{t+1}) F_{k}(k_{t},1)
\end{align}
\end{enumerate}
Social Planners Problem:  Sub the constraint in and take the FOC with respect to capital to directly obtain
\begin{align}
u_{c}(c_{t}) = \beta u(c_{t+1}) F_{k}(k_{t},1),
\end{align}
which is the same equilibrium condition as the competitive case, hence the allocatuions for $\{c, k, l\}$ are the same in both cases. (maybe write only in terms of k, then mention budget/ feasiblitly implies consumption is same in both cases)

%%%%%%%%%%%%%%%%%%%%%%%%%%%%%%%%%%%%%%%%%%%%%%%%%%%%%%%%%%%%%%%%%%%%%%%%%
\section{Question 4}
The planners dynamic programming problem is, 
\begin{align}
&V(k_{t}) = \underset{ k_{t+1}}{\text{max}} ~u(F(k_{t}, 1) - k_{t+1}) + \beta V(k_{t+1})\\
\text{ s.t. } &k_{t} \geq 0, F(k_{t}, 1) - k_{t+1} \geq 0,  k_{0} \text{ given},  
\end{align}

%%%%%%%%%%%%%%%%%%%%%%%%%%%%%%%%%%%%%%%%%%%%%%%%%%%%%%%%%%%%%%%%%%%%%%%%%
\section{Question 5}
\begin{align}
&V(k_{t}) = \underset{ k_{t+1}}{\text{max}}~ log(zk_{t}^{\alpha} - k_{t+1}) + \beta V(k_{t+1})\\
\text{ s.t. } &k_{t} \geq 0, zk^{\alpha}  - k_{t+1} \geq 0,  k_{0} \text{ given},  
\end{align}
Take FOC w.r.t $k_{t+1}$ to obtain
\begin{align}
d
\end{align}

\end{document}